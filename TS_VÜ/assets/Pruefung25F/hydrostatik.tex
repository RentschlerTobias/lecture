\aufgloes{gesamtPunkte_hydro}{
	%
	% Aufgabenstellung
	%
	\nuraufg{
		Eine hydraulische Hebevorrichtung der Tiefe $b$ besteht aus zwei masselosen Kolben, die sich in ihrem jeweiligen Hubraum ab oder auf bewegen k�nnen. Bef�llt ist die Vorrichtung mit einem �l der Dichte $\rho$. Zus�tzlich wirken auf die Kolben die Kr�fte $F_1$ und $F_2$ in negativer $y$-Richtung. Eine im Punkt $M$ drehbare Klappe �ffnet ab einem kritischen Moment $M_{krit}$. Ein W�rfel der Kantenl�nge $\frac{1}{2}H$ schwebt im �l.%
		\begin{center}
			\aufgloes{gesamtPunkte_hydro}{
	%
	% Aufgabenstellung
	%
	\nuraufg{
		Eine hydraulische Hebevorrichtung der Tiefe $b$ besteht aus zwei masselosen Kolben, die sich in ihrem jeweiligen Hubraum ab oder auf bewegen k�nnen. Bef�llt ist die Vorrichtung mit einem �l der Dichte $\rho$. Zus�tzlich wirken auf die Kolben die Kr�fte $F_1$ und $F_2$ in negativer $y$-Richtung. Eine im Punkt $M$ drehbare Klappe �ffnet ab einem kritischen Moment $M_{krit}$. Ein W�rfel der Kantenl�nge $\frac{1}{2}H$ schwebt im �l.%
		\begin{center}
			\aufgloes{gesamtPunkte_hydro}{
	%
	% Aufgabenstellung
	%
	\nuraufg{
		Eine hydraulische Hebevorrichtung der Tiefe $b$ besteht aus zwei masselosen Kolben, die sich in ihrem jeweiligen Hubraum ab oder auf bewegen k�nnen. Bef�llt ist die Vorrichtung mit einem �l der Dichte $\rho$. Zus�tzlich wirken auf die Kolben die Kr�fte $F_1$ und $F_2$ in negativer $y$-Richtung. Eine im Punkt $M$ drehbare Klappe �ffnet ab einem kritischen Moment $M_{krit}$. Ein W�rfel der Kantenl�nge $\frac{1}{2}H$ schwebt im �l.%
		\begin{center}
			\aufgloes{gesamtPunkte_hydro}{
	%
	% Aufgabenstellung
	%
	\nuraufg{
		Eine hydraulische Hebevorrichtung der Tiefe $b$ besteht aus zwei masselosen Kolben, die sich in ihrem jeweiligen Hubraum ab oder auf bewegen k�nnen. Bef�llt ist die Vorrichtung mit einem �l der Dichte $\rho$. Zus�tzlich wirken auf die Kolben die Kr�fte $F_1$ und $F_2$ in negativer $y$-Richtung. Eine im Punkt $M$ drehbare Klappe �ffnet ab einem kritischen Moment $M_{krit}$. Ein W�rfel der Kantenl�nge $\frac{1}{2}H$ schwebt im �l.%
		\begin{center}
			\input{hydrostatik.pdf_tex}
		\end{center}
		%
		\begin{annahmen}
			\item Die Fl�ssigkeit ist inkompressibel.
			\item Der sich jeweils einstellende Zustand ist station�r.
			\item Die Klappe schlie�t im geschlossenen Zustand dicht ab.
		\end{annahmen}
		
		%
		\begin{gegeben}
			& Kolbenfl�che: $A_1=A,A_2=4A$
			& H�he: $H$\\  
			& Umgebungsdruck: $p_0$
			& Erdbeschleunigung: $g$ \\  
			& Beh�lter- und Klappentiefe: $b$
			& Kritisches Moment: $M_{krit}$  \\
			& Dichte: $\rho$
		\end{gegeben}
	}
	%
	%    
	\nuraufg{
		Zun�chst sei $F_1 = F_2 = 0$.
	}
	
	
	
	%
	% a)
	% 
	\teil
	{
		Berechnen Sie die auf den W�rfel wirkende Auftriebskraft und dessen Dichte $\rho_w$.
	}
	{
		Berechnen der Auftriebskraft $F_{Auftrieb}$ auf den W�rfel. 
		\begin{equation*}
			F_{Auftrieb} = \frac{\rho g H^{3}}{8}  \; \teilpunkte{gesamtPunkte_hydro}{1}
		\end{equation*}
		Die Dichte $\rho_w$ entspricht der Dichte $\rho$ damit der W�rfel in �l schweben kann.  \; \teilpunkte{gesamtPunkte_hydro}{1}
		
	}
	% b)  
	\teil
	{
		Berechnen Sie den Betrag der resultierenden Druckkraft $F_{p}$ auf die Klappe.
	}
	{ 
		\begin{equation*}
			F_{p} = p_G A = \frac{3}{2} \rho g {H}^2 b\;
			\teilpunkte{gesamtPunkte_hydro}{2}
		\end{equation*}
		
	}
	%
	% c)
	%
	\teil
	{
		Berechnen Sie das von der Druckkraft $F_p$ erzeugte Moment um den Drehpunkt M.
	}
	{ Hebelarm:
		\begin{equation*}
			\begin{split}
				l_{p} & = \frac{H}{2} + \frac{\rho g}{p_G A} I_{\tilde{x}\tilde{x}}                 \\
				      & = \frac{H}{2} + \frac{\rho g}{\rho g \frac{3H}{2} H b} \frac{1}{12} {H}^3 b \\
				      & = \frac{5}{9} H                                                             \\
			\end{split}
			\teilpunkte{gesamtPunkte_hydro}{2}
		\end{equation*}
		
		
		Resultierendes Moment:
		\begin{equation*}
			\begin{split}
				M_p & = F_{p} l_{p}                             \\
				    & = \frac{3}{2} \rho g  H^2 b \frac{5}{9} H \\
				    & = \frac{5}{6} \rho g {H}^3 b \;
			\end{split}
			\teilpunkte{gesamtPunkte_hydro}{1}
		\end{equation*}
	}
	
	\nuraufg{
		Nun wirke die Kraft $F_1 \neq 0$. Die Positionen der Kolben und des W�rfels bleiben gleich.
	}
	%  d)
	\teil
	{
		Wie ver�ndern sich die auf die Ober- und Unterseite des W�rfels wirkenden Druckkr�fte und die Auftriebskraft?
	}
	{
		$F_{o}$ und $F_{u}$ sind gr��er. \teilpunkte{gesamtPunkte_hydro}{1}
		$F_{Auftrieb}$ bleibt gleich. \teilpunkte{gesamtPunkte_hydro}{1}
		
	}
	%  e)
	\teil
	{
		Skizzieren Sie den Druckverlauf �ber die linke und rechte Seite der Klappe und geben Sie die charakteristischen Werte an.
	}
	{
	
		Druckverlauf: \teilpunkte{gesamtPunkte_hydro}{3}
		
		\begin{center}
			\input{druckverlauf_teilaufgabe_c.pdf_tex}
		\end{center}
		
	}
	% g)
	%
	\teil
	{
		Berechnen Sie die Kraft $F_2$ in Abh�ngigkeit von $F_1$, damit das System im Ruhezustand ist.
	}
	{
		\begin{equation*}
			p_0+\frac{F_1}{A} = p_0+\frac{F_2}{4A}
		\end{equation*}
		
		\begin{equation*}
			F_2=4F_1
			\teilpunkte{gesamtPunkte_hydro}{1}
		\end{equation*}
		
	}
	
	
	%
	% g)
	%
	\teil
	{
		Bestimmen Sie die Kraft $F_1$, bei der sich die Klappe �ffnet.
	}
	{ Druckkraft:
		\begin{equation*}
			F_{p} = p_G A =\left( \frac{F_1}{A}+\frac{3}{2} \rho gH \right) Hb\;
			\teilpunkte{gesamtPunkte_hydro}{1}
		\end{equation*}
		Hebelarm:
		\begin{equation*}
			\begin{split}
				l_{p} & = \frac{H}{2} + \frac{\rho g}{p_G A} I_{\tilde{x}\tilde{x}}                                              \\
				      & = \frac{H}{2} + \frac{\rho g}{\left( \frac{F_1}{A}+\frac{3}{2} \rho gH \right) H b} \frac{1}{12} {H}^3 b \\
			\end{split}
			\teilpunkte{gesamtPunkte_hydro}{1}
		\end{equation*}
		Klappe �ffnet ab einem kritischen Moment $M_{krit}$:
		\begin{equation*}
			\left( \frac{F_1}{A}+\frac{3}{2} \rho gH \right) Hb \left( \frac{H}{2} + \frac{\rho g}{\left( \frac{F_1}{A}+\frac{3}{2} \rho gH \right) H b} \frac{1}{12} {H}^3 b \right) \overset{!}{=} M_{krit}
			\teilpunkte{gesamtPunkte_hydro}{2}
		\end{equation*}
		\begin{equation*}
			\left( \frac{F_1}{A}+\frac{3}{2} \rho gH \right) \frac{H^2 b}{2} + \frac{ \rho gH^3 b }{12}= M_{krit}
		\end{equation*}
		
		\begin{equation*}
			F_1=\frac{2A M_{krit}}{H^2 b}-\frac{5}{3}\rho gHA
			\teilpunkte{gesamtPunkte_hydro}{1}
		\end{equation*}
	}
}

		\end{center}
		%
		\begin{annahmen}
			\item Die Fl�ssigkeit ist inkompressibel.
			\item Der sich jeweils einstellende Zustand ist station�r.
			\item Die Klappe schlie�t im geschlossenen Zustand dicht ab.
		\end{annahmen}
		
		%
		\begin{gegeben}
			& Kolbenfl�che: $A_1=A,A_2=4A$
			& H�he: $H$\\  
			& Umgebungsdruck: $p_0$
			& Erdbeschleunigung: $g$ \\  
			& Beh�lter- und Klappentiefe: $b$
			& Kritisches Moment: $M_{krit}$  \\
			& Dichte: $\rho$
		\end{gegeben}
	}
	%
	%    
	\nuraufg{
		Zun�chst sei $F_1 = F_2 = 0$.
	}
	
	
	
	%
	% a)
	% 
	\teil
	{
		Berechnen Sie die auf den W�rfel wirkende Auftriebskraft und dessen Dichte $\rho_w$.
	}
	{
		Berechnen der Auftriebskraft $F_{Auftrieb}$ auf den W�rfel. 
		\begin{equation*}
			F_{Auftrieb} = \frac{\rho g H^{3}}{8}  \; \teilpunkte{gesamtPunkte_hydro}{1}
		\end{equation*}
		Die Dichte $\rho_w$ entspricht der Dichte $\rho$ damit der W�rfel in �l schweben kann.  \; \teilpunkte{gesamtPunkte_hydro}{1}
		
	}
	% b)  
	\teil
	{
		Berechnen Sie den Betrag der resultierenden Druckkraft $F_{p}$ auf die Klappe.
	}
	{ 
		\begin{equation*}
			F_{p} = p_G A = \frac{3}{2} \rho g {H}^2 b\;
			\teilpunkte{gesamtPunkte_hydro}{2}
		\end{equation*}
		
	}
	%
	% c)
	%
	\teil
	{
		Berechnen Sie das von der Druckkraft $F_p$ erzeugte Moment um den Drehpunkt M.
	}
	{ Hebelarm:
		\begin{equation*}
			\begin{split}
				l_{p} & = \frac{H}{2} + \frac{\rho g}{p_G A} I_{\tilde{x}\tilde{x}}                 \\
				      & = \frac{H}{2} + \frac{\rho g}{\rho g \frac{3H}{2} H b} \frac{1}{12} {H}^3 b \\
				      & = \frac{5}{9} H                                                             \\
			\end{split}
			\teilpunkte{gesamtPunkte_hydro}{2}
		\end{equation*}
		
		
		Resultierendes Moment:
		\begin{equation*}
			\begin{split}
				M_p & = F_{p} l_{p}                             \\
				    & = \frac{3}{2} \rho g  H^2 b \frac{5}{9} H \\
				    & = \frac{5}{6} \rho g {H}^3 b \;
			\end{split}
			\teilpunkte{gesamtPunkte_hydro}{1}
		\end{equation*}
	}
	
	\nuraufg{
		Nun wirke die Kraft $F_1 \neq 0$. Die Positionen der Kolben und des W�rfels bleiben gleich.
	}
	%  d)
	\teil
	{
		Wie ver�ndern sich die auf die Ober- und Unterseite des W�rfels wirkenden Druckkr�fte und die Auftriebskraft?
	}
	{
		$F_{o}$ und $F_{u}$ sind gr��er. \teilpunkte{gesamtPunkte_hydro}{1}
		$F_{Auftrieb}$ bleibt gleich. \teilpunkte{gesamtPunkte_hydro}{1}
		
	}
	%  e)
	\teil
	{
		Skizzieren Sie den Druckverlauf �ber die linke und rechte Seite der Klappe und geben Sie die charakteristischen Werte an.
	}
	{
	
		Druckverlauf: \teilpunkte{gesamtPunkte_hydro}{3}
		
		\begin{center}
			\input{druckverlauf_teilaufgabe_c.pdf_tex}
		\end{center}
		
	}
	% g)
	%
	\teil
	{
		Berechnen Sie die Kraft $F_2$ in Abh�ngigkeit von $F_1$, damit das System im Ruhezustand ist.
	}
	{
		\begin{equation*}
			p_0+\frac{F_1}{A} = p_0+\frac{F_2}{4A}
		\end{equation*}
		
		\begin{equation*}
			F_2=4F_1
			\teilpunkte{gesamtPunkte_hydro}{1}
		\end{equation*}
		
	}
	
	
	%
	% g)
	%
	\teil
	{
		Bestimmen Sie die Kraft $F_1$, bei der sich die Klappe �ffnet.
	}
	{ Druckkraft:
		\begin{equation*}
			F_{p} = p_G A =\left( \frac{F_1}{A}+\frac{3}{2} \rho gH \right) Hb\;
			\teilpunkte{gesamtPunkte_hydro}{1}
		\end{equation*}
		Hebelarm:
		\begin{equation*}
			\begin{split}
				l_{p} & = \frac{H}{2} + \frac{\rho g}{p_G A} I_{\tilde{x}\tilde{x}}                                              \\
				      & = \frac{H}{2} + \frac{\rho g}{\left( \frac{F_1}{A}+\frac{3}{2} \rho gH \right) H b} \frac{1}{12} {H}^3 b \\
			\end{split}
			\teilpunkte{gesamtPunkte_hydro}{1}
		\end{equation*}
		Klappe �ffnet ab einem kritischen Moment $M_{krit}$:
		\begin{equation*}
			\left( \frac{F_1}{A}+\frac{3}{2} \rho gH \right) Hb \left( \frac{H}{2} + \frac{\rho g}{\left( \frac{F_1}{A}+\frac{3}{2} \rho gH \right) H b} \frac{1}{12} {H}^3 b \right) \overset{!}{=} M_{krit}
			\teilpunkte{gesamtPunkte_hydro}{2}
		\end{equation*}
		\begin{equation*}
			\left( \frac{F_1}{A}+\frac{3}{2} \rho gH \right) \frac{H^2 b}{2} + \frac{ \rho gH^3 b }{12}= M_{krit}
		\end{equation*}
		
		\begin{equation*}
			F_1=\frac{2A M_{krit}}{H^2 b}-\frac{5}{3}\rho gHA
			\teilpunkte{gesamtPunkte_hydro}{1}
		\end{equation*}
	}
}

		\end{center}
		%
		\begin{annahmen}
			\item Die Fl�ssigkeit ist inkompressibel.
			\item Der sich jeweils einstellende Zustand ist station�r.
			\item Die Klappe schlie�t im geschlossenen Zustand dicht ab.
		\end{annahmen}
		
		%
		\begin{gegeben}
			& Kolbenfl�che: $A_1=A,A_2=4A$
			& H�he: $H$\\  
			& Umgebungsdruck: $p_0$
			& Erdbeschleunigung: $g$ \\  
			& Beh�lter- und Klappentiefe: $b$
			& Kritisches Moment: $M_{krit}$  \\
			& Dichte: $\rho$
		\end{gegeben}
	}
	%
	%    
	\nuraufg{
		Zun�chst sei $F_1 = F_2 = 0$.
	}
	
	
	
	%
	% a)
	% 
	\teil
	{
		Berechnen Sie die auf den W�rfel wirkende Auftriebskraft und dessen Dichte $\rho_w$.
	}
	{
		Berechnen der Auftriebskraft $F_{Auftrieb}$ auf den W�rfel. 
		\begin{equation*}
			F_{Auftrieb} = \frac{\rho g H^{3}}{8}  \; \teilpunkte{gesamtPunkte_hydro}{1}
		\end{equation*}
		Die Dichte $\rho_w$ entspricht der Dichte $\rho$ damit der W�rfel in �l schweben kann.  \; \teilpunkte{gesamtPunkte_hydro}{1}
		
	}
	% b)  
	\teil
	{
		Berechnen Sie den Betrag der resultierenden Druckkraft $F_{p}$ auf die Klappe.
	}
	{ 
		\begin{equation*}
			F_{p} = p_G A = \frac{3}{2} \rho g {H}^2 b\;
			\teilpunkte{gesamtPunkte_hydro}{2}
		\end{equation*}
		
	}
	%
	% c)
	%
	\teil
	{
		Berechnen Sie das von der Druckkraft $F_p$ erzeugte Moment um den Drehpunkt M.
	}
	{ Hebelarm:
		\begin{equation*}
			\begin{split}
				l_{p} & = \frac{H}{2} + \frac{\rho g}{p_G A} I_{\tilde{x}\tilde{x}}                 \\
				      & = \frac{H}{2} + \frac{\rho g}{\rho g \frac{3H}{2} H b} \frac{1}{12} {H}^3 b \\
				      & = \frac{5}{9} H                                                             \\
			\end{split}
			\teilpunkte{gesamtPunkte_hydro}{2}
		\end{equation*}
		
		
		Resultierendes Moment:
		\begin{equation*}
			\begin{split}
				M_p & = F_{p} l_{p}                             \\
				    & = \frac{3}{2} \rho g  H^2 b \frac{5}{9} H \\
				    & = \frac{5}{6} \rho g {H}^3 b \;
			\end{split}
			\teilpunkte{gesamtPunkte_hydro}{1}
		\end{equation*}
	}
	
	\nuraufg{
		Nun wirke die Kraft $F_1 \neq 0$. Die Positionen der Kolben und des W�rfels bleiben gleich.
	}
	%  d)
	\teil
	{
		Wie ver�ndern sich die auf die Ober- und Unterseite des W�rfels wirkenden Druckkr�fte und die Auftriebskraft?
	}
	{
		$F_{o}$ und $F_{u}$ sind gr��er. \teilpunkte{gesamtPunkte_hydro}{1}
		$F_{Auftrieb}$ bleibt gleich. \teilpunkte{gesamtPunkte_hydro}{1}
		
	}
	%  e)
	\teil
	{
		Skizzieren Sie den Druckverlauf �ber die linke und rechte Seite der Klappe und geben Sie die charakteristischen Werte an.
	}
	{
	
		Druckverlauf: \teilpunkte{gesamtPunkte_hydro}{3}
		
		\begin{center}
			\input{druckverlauf_teilaufgabe_c.pdf_tex}
		\end{center}
		
	}
	% g)
	%
	\teil
	{
		Berechnen Sie die Kraft $F_2$ in Abh�ngigkeit von $F_1$, damit das System im Ruhezustand ist.
	}
	{
		\begin{equation*}
			p_0+\frac{F_1}{A} = p_0+\frac{F_2}{4A}
		\end{equation*}
		
		\begin{equation*}
			F_2=4F_1
			\teilpunkte{gesamtPunkte_hydro}{1}
		\end{equation*}
		
	}
	
	
	%
	% g)
	%
	\teil
	{
		Bestimmen Sie die Kraft $F_1$, bei der sich die Klappe �ffnet.
	}
	{ Druckkraft:
		\begin{equation*}
			F_{p} = p_G A =\left( \frac{F_1}{A}+\frac{3}{2} \rho gH \right) Hb\;
			\teilpunkte{gesamtPunkte_hydro}{1}
		\end{equation*}
		Hebelarm:
		\begin{equation*}
			\begin{split}
				l_{p} & = \frac{H}{2} + \frac{\rho g}{p_G A} I_{\tilde{x}\tilde{x}}                                              \\
				      & = \frac{H}{2} + \frac{\rho g}{\left( \frac{F_1}{A}+\frac{3}{2} \rho gH \right) H b} \frac{1}{12} {H}^3 b \\
			\end{split}
			\teilpunkte{gesamtPunkte_hydro}{1}
		\end{equation*}
		Klappe �ffnet ab einem kritischen Moment $M_{krit}$:
		\begin{equation*}
			\left( \frac{F_1}{A}+\frac{3}{2} \rho gH \right) Hb \left( \frac{H}{2} + \frac{\rho g}{\left( \frac{F_1}{A}+\frac{3}{2} \rho gH \right) H b} \frac{1}{12} {H}^3 b \right) \overset{!}{=} M_{krit}
			\teilpunkte{gesamtPunkte_hydro}{2}
		\end{equation*}
		\begin{equation*}
			\left( \frac{F_1}{A}+\frac{3}{2} \rho gH \right) \frac{H^2 b}{2} + \frac{ \rho gH^3 b }{12}= M_{krit}
		\end{equation*}
		
		\begin{equation*}
			F_1=\frac{2A M_{krit}}{H^2 b}-\frac{5}{3}\rho gHA
			\teilpunkte{gesamtPunkte_hydro}{1}
		\end{equation*}
	}
}

		\end{center}
		%
		\begin{annahmen}
			\item Die Fl�ssigkeit ist inkompressibel.
			\item Der sich jeweils einstellende Zustand ist station�r.
			\item Die Klappe schlie�t im geschlossenen Zustand dicht ab.
		\end{annahmen}
		
		%
		\begin{gegeben}
			& Kolbenfl�che: $A_1=A,A_2=4A$
			& H�he: $H$\\  
			& Umgebungsdruck: $p_0$
			& Erdbeschleunigung: $g$ \\  
			& Beh�lter- und Klappentiefe: $b$
			& Kritisches Moment: $M_{krit}$  \\
			& Dichte: $\rho$
		\end{gegeben}
	}
	%
	%    
	\nuraufg{
		Zun�chst sei $F_1 = F_2 = 0$.
	}
	
	
	
	%
	% a)
	% 
	\teil
	{
		Berechnen Sie die auf den W�rfel wirkende Auftriebskraft und dessen Dichte $\rho_w$.
	}
	{
		Berechnen der Auftriebskraft $F_{Auftrieb}$ auf den W�rfel. 
		\begin{equation*}
			F_{Auftrieb} = \frac{\rho g H^{3}}{8}  \; \teilpunkte{gesamtPunkte_hydro}{1}
		\end{equation*}
		Die Dichte $\rho_w$ entspricht der Dichte $\rho$ damit der W�rfel in �l schweben kann.  \; \teilpunkte{gesamtPunkte_hydro}{1}
		
	}
	% b)  
	\teil
	{
		Berechnen Sie den Betrag der resultierenden Druckkraft $F_{p}$ auf die Klappe.
	}
	{ 
		\begin{equation*}
			F_{p} = p_G A = \frac{3}{2} \rho g {H}^2 b\;
			\teilpunkte{gesamtPunkte_hydro}{2}
		\end{equation*}
		
	}
	%
	% c)
	%
	\teil
	{
		Berechnen Sie das von der Druckkraft $F_p$ erzeugte Moment um den Drehpunkt M.
	}
	{ Hebelarm:
		\begin{equation*}
			\begin{split}
				l_{p} & = \frac{H}{2} + \frac{\rho g}{p_G A} I_{\tilde{x}\tilde{x}}                 \\
				      & = \frac{H}{2} + \frac{\rho g}{\rho g \frac{3H}{2} H b} \frac{1}{12} {H}^3 b \\
				      & = \frac{5}{9} H                                                             \\
			\end{split}
			\teilpunkte{gesamtPunkte_hydro}{2}
		\end{equation*}
		
		
		Resultierendes Moment:
		\begin{equation*}
			\begin{split}
				M_p & = F_{p} l_{p}                             \\
				    & = \frac{3}{2} \rho g  H^2 b \frac{5}{9} H \\
				    & = \frac{5}{6} \rho g {H}^3 b \;
			\end{split}
			\teilpunkte{gesamtPunkte_hydro}{1}
		\end{equation*}
	}
	
	\nuraufg{
		Nun wirke die Kraft $F_1 \neq 0$. Die Positionen der Kolben und des W�rfels bleiben gleich.
	}
	%  d)
	\teil
	{
		Wie ver�ndern sich die auf die Ober- und Unterseite des W�rfels wirkenden Druckkr�fte und die Auftriebskraft?
	}
	{
		$F_{o}$ und $F_{u}$ sind gr��er. \teilpunkte{gesamtPunkte_hydro}{1}
		$F_{Auftrieb}$ bleibt gleich. \teilpunkte{gesamtPunkte_hydro}{1}
		
	}
	%  e)
	\teil
	{
		Skizzieren Sie den Druckverlauf �ber die linke und rechte Seite der Klappe und geben Sie die charakteristischen Werte an.
	}
	{
	
		Druckverlauf: \teilpunkte{gesamtPunkte_hydro}{3}
		
		\begin{center}
			\input{druckverlauf_teilaufgabe_c.pdf_tex}
		\end{center}
		
	}
	% g)
	%
	\teil
	{
		Berechnen Sie die Kraft $F_2$ in Abh�ngigkeit von $F_1$, damit das System im Ruhezustand ist.
	}
	{
		\begin{equation*}
			p_0+\frac{F_1}{A} = p_0+\frac{F_2}{4A}
		\end{equation*}
		
		\begin{equation*}
			F_2=4F_1
			\teilpunkte{gesamtPunkte_hydro}{1}
		\end{equation*}
		
	}
	
	
	%
	% g)
	%
	\teil
	{
		Bestimmen Sie die Kraft $F_1$, bei der sich die Klappe �ffnet.
	}
	{ Druckkraft:
		\begin{equation*}
			F_{p} = p_G A =\left( \frac{F_1}{A}+\frac{3}{2} \rho gH \right) Hb\;
			\teilpunkte{gesamtPunkte_hydro}{1}
		\end{equation*}
		Hebelarm:
		\begin{equation*}
			\begin{split}
				l_{p} & = \frac{H}{2} + \frac{\rho g}{p_G A} I_{\tilde{x}\tilde{x}}                                              \\
				      & = \frac{H}{2} + \frac{\rho g}{\left( \frac{F_1}{A}+\frac{3}{2} \rho gH \right) H b} \frac{1}{12} {H}^3 b \\
			\end{split}
			\teilpunkte{gesamtPunkte_hydro}{1}
		\end{equation*}
		Klappe �ffnet ab einem kritischen Moment $M_{krit}$:
		\begin{equation*}
			\left( \frac{F_1}{A}+\frac{3}{2} \rho gH \right) Hb \left( \frac{H}{2} + \frac{\rho g}{\left( \frac{F_1}{A}+\frac{3}{2} \rho gH \right) H b} \frac{1}{12} {H}^3 b \right) \overset{!}{=} M_{krit}
			\teilpunkte{gesamtPunkte_hydro}{2}
		\end{equation*}
		\begin{equation*}
			\left( \frac{F_1}{A}+\frac{3}{2} \rho gH \right) \frac{H^2 b}{2} + \frac{ \rho gH^3 b }{12}= M_{krit}
		\end{equation*}
		
		\begin{equation*}
			F_1=\frac{2A M_{krit}}{H^2 b}-\frac{5}{3}\rho gHA
			\teilpunkte{gesamtPunkte_hydro}{1}
		\end{equation*}
	}
}
